%   Filename    : abstract.tex 
\begin{center}
\textbf{Abstract}
\end{center}
\setlength{\parindent}{0pt}
Road surveying is a crucial part of the maintenance processes of roads in the Philippines that is carried out by the Department of Public Works and Highways. However, the current process of road surveying is time consuming which delays much needed maintenance operations. Existing studies involving automated pothole detection lack integration of the pothole's depth in assessing its severity which is essential for automating road  surveying procedures.  A system that incorporates estimated depth information in assessing pothole severity is developed in order to automate the manual process of depth measurement and severity assessment in road surveying. For depth estimation, stereo vision is favorable in this context as depth may be estimated through the disparity generated by a stereo pair. In obtaining a stereo view of the potholes, the StereoPi V2 is utilized along with some modifications that would make it eligible for outdoor use.
To address camera imperfections, a fitted inverse model was applied to improve the accuracy of depth estimates. Linear regression analysis revealed a strong positive correlation (R = 0.978) between estimated and actual depths, with the system measuring pothole depths mostly within 3 cm of the true values.

%  Do not put citations or quotes in the abract.

\begin{tabular}{lp{4.25in}}
\hspace{-0.5em}\textbf{Keywords:}\hspace{0.25em} & pothole, depth estimation, stereo vision, StereoPi V2\\
\end{tabular}
