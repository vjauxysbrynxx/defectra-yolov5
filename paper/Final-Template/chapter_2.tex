%   Filename    : chapter_2.tex 
\chapter{Review of Related Literature}

\section{Frameworks}
This section of the chapter presents related frameworks that is considered essential for the development of this special problem.

\subsection{Depth Estimation}
Depth estimation as defined by \citeA{sanz2012} is a set of processes that aims to extract a representation of a certain scene's spatial composition. Stereo vision is stated to be among the depth estimation strategies.

\subsection{Image and Video Processing}
\citeA{kumar2024} defines image processing as a process of turning an image into its digital form and extracting data from it through certain functions and operations. Usual processes are considered to treat images as 2D signals wherein different processing methods utilize these signals.
Like image processing, \citeA{riches2014} defines video processing as being able to extract information and data from video footage through signal processing methods. However, in video processing due to the diversity of video formats, compression and decompression methods are often expected to be performed on videos before processing methods to either increase or decrease bitrate.

\subsection{Stereo Vision}
MathWorks (n.d.) defines stereo vision as a process of utilizing multiple 2D perspectives in order to extract information in 3D. In addition, most uses of stereo vision involve estimating an objects distance from an observer or camera. The 3D information is stated to be extracted with stereo pairs or pair of images through estimation of relative depth of points in a scene which are then represented through a stereo map that is made through the matching of the pair's corresponding points.


\section{Related Studies}
This section of the chapter presents related studies conducted by other researchers wherein the methodology and technologies used may serve as basis in the development of this special problem.

\subsection{Deep Learning Studies}

\noindent\textbf{\large Automated Detection and Classification of Road Anomalies in VANET Using Deep Learning} \\\\
In the study of Bibi et al. (2021) it was noted that identification of active road defects are critical in maintaining smooth and safe flow of traffic. Detection and subsequent repair of such defects in roads are crucial in keeping vehicles using such roads away from mechanical failures. The study also emphasized the growth in use of autonomous vehicles in research data gathering which is what the researchers utilized in data gathering procedures. With the presence of autonomous vehicles, this allowed the researchers to use a combination of sensors and deep neural networks in deploying artificial intelligence. The study aimed to allow autonomous vehicles to avoid critical road defects that can possibly lead to dangerous situations. Researchers used Resnet-18 and VGG-11 in automatic detection and classification of road defects. Researchers concluded that the trained model was able to perform better than other techniques for road defect detection. The study is able to provide the effectiveness in automating road defect detection and classification. However, the study lacks findings regarding the severity of detected defects and incorporation of pothole depth in their model which are both crucial in automating manual procedures of road surveying in the Philippines.

\noindent\textbf{\large Single Image Depth Estimation: An Overview} \\\\
In the study by \citeA{mertan2022}, the authors argued that machine learning methods, specifically convolutional neural networks (CNNs), are among the most effective approaches for solving the depth estimation problem. They noted that most existing depth estimation studies address this task by utilizing relative depth information derived from labeled datasets. Additionally, visual cues such as ground plane contact, vanishing points, and object edges were identified as key features for estimating depth from a single image.
The researchers also pointed out that relying on labeled data may introduce biases, which can affect the accuracy of these learned cues. While the limitations of single-image depth estimation were acknowledged, the study did not thoroughly explore alternative methods such as stereo imaging, which can produce more precise depth maps and potentially address some of these limitations.

\noindent\textbf{\large Assessing Severity of Road Cracks Using Deep Learning-Based Segmentation and Detection} \\\\
In the study of \citeA{ha2022}, it was argued that the detection, classification, and severity assessment of road cracks should be automated due to the bottleneck it causes during the entire process of surveying. For the study, the researchers utilized SqueezNet, U-Net, and MobileNet-SSD models for crack classification and severity assessment. Furthermore, the researchers also employed separate U-nets for linear and area cracking cases. For crack detection, the researchers followed the process of pre-processing, detection, classification. During preprocessing images were smoothed out using image processing techniques. The researchers also utilized YOLOv5 object detection models for classification of pavement cracking wherein the YOLOv51 model recorded the highest accuracy. The researchers however stated images used for the study are only 2D images which may have allowed higher accuracy rates. Furthermore, the researchers suggest incorporating depth information in the models to further enhance results. Despite the accuracy of the deep learning models in identification and classification of road cracks, the lack of depth estimation and severity assessment suggests that the study is still not geared towards road surveying processes wherein depth estimation with severity assessment of individually detected road cracks may be required.

\subsection{Machine Learning Studies}

\noindent\textbf{\large Smartphones as Sensors for Road Surface Monitoring} \\\\
In their study, \citeA{sattar2018} noted the rise of sensing capabilities of smartphones which they utilized in monitoring road surface to detect and identify anomalies. The researchers considered different approaches in detecting road surface anomalies using smartphone sensors. One of which are threshold-based approaches which was determined to be quite difficult  due to several factors that are affecting the process of determining the interval length of a window function in spectral analysis. The researchers also utilized a machine learning approach adapted from another study. It was stated that k-means was used in classifying sensor data and in training the SVM algorithm. Due to the requirement of training a supervised algorithm using a labeled sample data was required before classifying data from sensors, the approach was considered to be impractical for real-time situations. In addition, \citeA{sattar2018} also noted various challenges when utilizing smartphones as sensors for data gathering such as sensors being dependent on the device’s placement and orientation, smoothness of captured data, and the speed of the vehicle it is being mounted on. Lastly, it was also concluded that the accuracy and performance of using smartphone sensors is challenging to compare due to the limited data sets and reported algorithms. With the smartphone's observed limitations in surveying road conditions, this indicates that much more sophisticated imaging technologies may be utilized in realtime surveying procedures. In addition, the smartphone's over reliance on several factors also makes it quite incapable in accurate depth estimation.

\subsection{Computer Vision Studies}

\noindent\textbf{\large Stereo Vision Based Pothole Detection System for Improved Ride Quality} \\\\
In the study of \citeA{ramaiah2021} it was stated that stereo vision has been earning attention due to its reliable obstacle detection and recognition. Furthermore, the study also discussed that such technology would be useful in improving ride quality in automated vehicles by integrating it in a predictive suspension control system. The proposed study was to develop a novel stereo vision based pothole detection system which also calculates the depth accurately. However, the study focused on improving ride quality by using the 3D information from detected potholes in controlling the damping coefficient of the suspension system. Overall, the pothole detection system was able to achieve 84\% accuracy and is able to detect potholes that are deeper than 5 cm. The researchers concluded that such system can be utilized in commercial applications. However, it is also worth noting that despite the system being able to detect potholes and measure its depth, the overall severity of the pothole and road condition was not addressed which makes it quite inapplicable for automated road surveying purposes. \\

\noindent\textbf{\large Depth and Image Fusion for Road Obstacle Detection Using Stereo Camera} \\\\
In the study of \citeA{perezyabov2022}, the researchers utilized stereo imaging in detecting obstacles in the road as well as their distance from the camera through the use of depth information gathered from the stereo cameras. It was stated that obstacle detection was a challenge due to certain factors such as artificial illumination and various road textures. In order to address these limitations, the researchers developed an RGB-based and obstacle detection stereo-based approach where SLIC superpixel segmentation was integrated for object segmentation. The findings were reported to give encouraging results due to the researchers being able to prove that RGB-based methods were capable of searching small contrasts objects making road obstacle detection possible. However, it was noted that significant background noise was visibile in their captures which may affect a detected obstacle's accuracy. In addition, due to this limitation, RGB-based methods for stereo image depth estimation may not produce accurate results. Furthermore, the researchers were only able to test such model in a parking lot wherein vehicle movement is slow and obstacles are almost easily recognizable, lack of testing in actual roads may indicate the model's unreadiness in an actual road applications.

\newpage
\section{Synthesis}
In majority of the studies discussed, road defect detection and classification is a common point of discussion. However, despite deep learning approaches being succesful in solving the problem of road defect detection, most of the studies still lack depth incorporation in their models which is considered as a factor in assessing pothole depth as based on the Long Term Pavement Performance \cite{miller2014}. Furthermore, for stereo vision studies, the detection aspect is also addressed however the studies are not geared towards road surveying processes due to the emphasis on driver and ride quality improvement. With the observed limitations in related studies, the researchers of this study focused on incorporating severity assessment with depth estimation through a stereo vision based approach to be able to build a foundation on depth based severity assessment that could be integrated in future deep learning models.

\newpage
\section{Chapter Summary}
The reviewed literature involved various techniques and approaches in  road anomaly detection and classification. These approaches are discussed and summarized below along with their limitations and research gaps.
\begin{table}[h!]
	\centering
	\hspace{-2cm}
	\small 
	\begin{tabular}{|p{3cm}|p{3.5cm}|p{4.5cm}|p{4.5cm}|}
		\hline
		\textbf{Study} & \textbf{Technology/
			Techniques Used} & \textbf{Key Findings} & \textbf{Limitations} \\ \hline
		
		Automated Detection and Classification of Road Anomalies in VANET Using Deep Learning & Resnet-18 and VGG-11 & Trained model is able to provide the effectiveness of using deep learning models in training artificial intelligence for road defect detection and classification. & Lacks findings regarding the severity of detected defects. \\ \hline
		
		Depth and Image Fusion for Road Obstacle Detection Using Stereo Camera & Stereo Imaging, RGB-based method & Model was able to take advantage of small contrast objects and detect obstacles. &  Approach was conducted in a controlled setting with inadequate practical application. \\ \hline
		
		Single Image Depth Estimation: An Overview & Deep Learning Models & Identified various issues with single image depth estimation and effective deep learning model approaches in solving the problem. & Other alternatives to depth estimation with respect to the limitations of single image depth estimation was not mentioned or thoroughly discussed. \\ \hline
		
		Assessing Severity of Road Cracks Using Deep Learning-Based Segmentation and Detection & SqueezNet, U-Net, YOLOv5, and MobileNet-SSD models & YOLOv51 model recorded the highest accuracy. & Only 2D images are used for the study which may have allowed higher accuracy rates, and the study also lacked depth information. \\ \hline
		
		Stereo Vision Based Pothole Detection System for Improved Ride Quality & Pair of stereo images captured by a stereo camera & System was able to achieve 84\% accuracy and is able to detect potholes that are deeper than 5 cm. & Overall severity of the pothole and road condition was not addressed. \\ \hline
		
	\end{tabular}
	\caption{Comparison of Related Studies on Road Anomaly Detection using Deep Learning Techniques and Stereo Vision}
	\label{tab:comparison}
	\hspace{-3cm}
	
	
\end{table}