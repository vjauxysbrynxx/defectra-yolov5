%   Filename    : chapter_4.tex 
\chapter{Methodology}
This chapter outlines the systematic approach that will be taken to address the problem of classifying and assessing road defects using artificial intelligence. The methodology will be divided into key phases: data collection, algorithm selection, design, testing and experimentation, and challenges and limitations. Each phase will play a crucial role in accurately classifying and assessing road defects.  Each phase will be essential for accurately classifying and assessing road defects. 

\section{\textbf{ Research Activities} }

\subsection{\textbf{Data Collection} }
The researchers conducted initial inquiries to understand the problem domain and existing road maintenance practices. This phase included consulting the engineers under the Road Maintenance Department of the government agency Department of Public Works and Highways (DPWH). An interview with Engr. Jane Chua provided a comprehensive overview of the DPWH's road maintenance manual, which was crucial in aligning this project with existing standards. This collaboration with DPWH provided insights into road pothole classification standards, ensuring that the collected data will align with industry standards. The researchers will also manually annotate the pilot dataset based on these standards, ensuring local relevance. 


\subsection{\textbf{Algorithm Selection } }
Potential solutions, algorithms, and system architectures were discussed by the researchers and the special problem adviser in this phase. These sessions, conducted in class and virtually via Zoom, helped narrow down the overview of the system, leading to the selection of the main architecture YOLOv5 for pothole detection and Epipolar Spatio-Temporal Networks (ESTN) for depth estimation. 

\subsubsection{Pothole Detection}
YOLOv5 was selected due to its high accuracy and ability to process images in real-time, making it suitable for detecting road defects in dynamic environments. Its architecture is optimized for speed and performance, which is crucial for large-scale deployment in road inspections. 

\subsubsection{Severity Assessment}
The Multi-view Depth Estimation using Epipolar Spatio-Temporal Networks was selected due to the high cost and limited accessibility of LiDAR technology. By applying epipolar geometry and temporal consistency across sequential frames, this approach provides an accurate depth estimation from standard video footage \cite{long2021}. 

\subsection{\textbf{Design, Testing, and Experimentation} }
This section outlines both the design and testing of the system, as well as the experimentation process to validate the selected methodologies. 

\subsubsection{Model Design}
The system was designed to operate with two core components: YOLOv5 for pothole detection and ESTN for severity assessment. The model architecture was chosen based on the real-time processing capabilities and the need for accurate depth estimation from standard video footage. The design ensures that the system can detect defects and provide severity assessments in a seamless workflow. 

\subsubsection{Data Set}
The YOLOv5 model was trained using two datasets from Universe Roboflow. One of the data sets was posted by a user named Eric Tam. It was also stated that the images from the dataset are sourced from a Crowdensing-based Road Damage Detection Challenge from 2022 in Japan. The challenge involves contestants being required to submit road damage datasets, shortlist their data set, and use the data set for road damage detection and classification models. The use of this data set in training models for road damage detection and classification ensures that the data is viable for training the YOLOv5 model. The dataset contains various road defects in Japan.
Another data set used in training the YOLOv5 model was also uploaded in Universe Roboflow by a user named Atikur Rahman Chitholian which was stated to be part of his undergraduate thesis. The dataset is comprised of 665 images with potholes being labeled. It was also stated that the data set can be utilized in automatically detecting and categorizing potholes found in the streets of cities.
Data preprocessing techniques were applied to both datasets to improve model accuracy and generalization. These included resizing images to a uniform size, applying augmentation techniques (flipping, rotation, and color adjustment) to increase dataset variability, and normalizing pixel values to ensure consistency across images. 

\subsubsection{Performance Metrics}
The performance of the YOLOv5 model will be evaluated using mean Average Precision (mAP). mAP is a widely used metric in object detection tasks and is particularly useful for assessing models that need to detect and classify multiple object categories. In this case, mAP will provide a comprehensive evaluation of the model's ability to detect and classify potholes, offering an aggregated score across the relevant detection thresholds. This ensures a balanced assessment of both detection accuracy and classification performance, which is essential for accurately identifying potholes across varying conditions. The effectiveness of mAP for this task is well-established in object detection literature (Everingham et al., 2015; Lin et al., 2014).

For the accuracy of depth estimation using the Epipolar Spatio-Temporal Networks (ESTN), Root Mean Squared Error (RMSE) and Mean Absolute Error (MAE) will be used. RMSE is chosen for its ability to penalize larger errors more heavily, making it suitable for assessing depth estimation performance where larger deviations from the ground truth are more significant (Zhang et al., 2018). MAE is also employed to provide a straightforward measure of average error magnitude, offering a complementary evaluation of depth estimation without emphasizing larger errors as much (Zhang et al., 2020).





\subsubsection{Testing and Validation}
The testing process will begin with a detailed testing plan that includes both simulated and real-world testing scenarios. Initially, the model will be tested in controlled environments to ensure it can detect and assess road defects accurately. Following this, real-world testing will be conducted using the StereoPi kit on local roads, specifically at the University of the Philippines Visayas Miagao Campus. The system's performance will be validated by comparing its predictions with ground-truth data collected from manual inspections. 

\subsubsection{Documentation}
Throughout the research activities, thorough documentation will be maintained. This documentation will capture all methods, results, challenges, and adjustments made during the experimentation phases. It ensures the reproducibility of the work and provides transparency for future research endeavors. 

\subsection{\textbf{Challenges and Limitations} }
\subsubsection{Availability of Local Datasets}
The lack of locally labeled datasets for road defects has posed a challenge in training accurate models. The majority of available datasets are sourced from international locations, which may not fully represent the road conditions found in the study area. To address the lack of locally labeled datasets, the researchers will create a pilot dataset from local roads within the University of the Philippines Visayas Miagao Campus. This dataset will be manually annotated according to DPWH's classification standards, ensuring local relevance.

\subsubsection{Data Quality and Variability}
Variations in the quality and resolution of the data collected from different sources may impact the performance of the trained models. In particular, images captured under varying weather conditions or lighting may affect the accuracy of pothole detection. To address this, the researchers plan to use the StereoPi kit to capture images under optimal weather and lighting conditions, such as mid-morning or early afternoon on clear days, ensuring consistent image quality for stereo vision analysis. The kit’s stereo cameras will be calibrated for uniform resolution and focus. Data augmentation techniques will also be applied to simulate varying conditions, and pre-processing steps like noise reduction and contrast enhancement will be used to improve the quality of the captured data. This approach aims to minimize the impact of environmental factors on the accuracy of road pothole detection and depth estimation.


\section{Calendar of Activities}
Table 1 shows a Gantt chart of the activities. Each bullet represents approximately one week's worth of activity.
%
%  the following commands will be used for filling up the bullets in the Gantt chart
%
%\newcommand{\weekone}{\textbullet}
%\newcommand{\weektwo}{\textbullet \textbullet}
%\newcommand{\weekthree}{\textbullet \textbullet \textbullet}
%\newcommand{\weekfour}{\textbullet \textbullet \textbullet \textbullet}

%
%  alternative to bullet is a star 
%
\begin{comment}
   \newcommand{\weekone}{$\star$}
   \newcommand{\weektwo}{$\star \star$}
   \newcommand{\weekthree}{$\star \star \star$}
   \newcommand{\weekfour}{$\star \star \star \star$ }
\end{comment}



\begin{table}[ht]   %t means place on top, replace with b if you want to place at the bottom
\centering
\caption{Timetable of Activities for 2024} \vspace{0.25em}
\begin{tabular}{|p{2in}|c|c|c|c|c|} \hline
\centering Activities (2024) & Aug   & Sept & Oct & Nov & Dec  \\ \hline
Pre-proposal Preparation      &  \weekfour     &     &  &  &   \\ \hline
Literature Review & ~~~\weekthree  & \weekone  & &  &    \\ \hline
Data Collection     &  \weektwo & \weektwo  &  &  &  \\ \hline
Algorithm Selection     &   & \weektwo &  &  &  \\ \hline
System Design      &   & \weekone  & \weektwo & ~~~\weektwo &    \\ \hline
Preliminary Testing &   &  &  & \weektwo & \weekone  \\ \hline
Documentation and SP Writing & ~~~\weekfour & \weekfour & \weekfour & \weekfour & \weektwo \\ \hline
\end{tabular}
\label{tab:timetableactivities}
\end{table}

\begin{table}[ht]   %t means place on top, replace with b if you want to place at the bottom
	\centering
	\caption{Timetable of Activities for 2025} \vspace{0.25em}
	\begin{tabular}{|p{2in}|c|c|c|c|c|c|} \hline
		\centering Activities (2025) & Jan   & Feb & Mar & Apr & May & Jun  \\ \hline
		Data Collection      &  \weekfour     &     &  &  & &  \\ \hline
		System Design & ~~~\weekthree  & \weektwo  & \weektwo &  & &   \\ \hline
		Model testing     &  \weekthree & \weekfour  &  \weekfour &  &  & \\ \hline
		Results Analysis    &   &  &  \weektwo & \weekfour& & \\ \hline
		Conclusion Formulation     &   &   &  & ~~~\weektwo &  ~~~\weekthree &  \\ \hline
		Documentation and SP Writing & ~~~\weekfour & \weekfour & \weekfour & \weekfour & \weekfour & \weektwo\\ \hline
	\end{tabular}
	\label{tab:timetableactivities}
\end{table}


