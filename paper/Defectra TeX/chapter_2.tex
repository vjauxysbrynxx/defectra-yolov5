%   Filename    : chapter_2.tex 
\chapter{Review of Related Literature}

\section{Frameworks}
This section of the chapter presents related frameworks that is considered essential for the development of this special problem.

\subsection{Depth Estimation}
Depth estimation as defined by \citeA{sanz2012} is a set of processes that aims to extract a representation of a certain scene's spatial composition. Stereo vision is stated to be among the depth estimation strategies.

\subsection{Image and Video Processing}
\citeA{kumar2024} defines image processing as a process of turning an image into its digital form and extracting data from it through certain functions and operations. Usual processes are considered to treat images as 2D signals wherein different processing methods utilize these signals.
Like image processing, \citeA{riches2014} defines video processing as being able to extract information and data from video footage through signal processing methods. However, in video processing due to the diversity of video formats, compression and decompression methods are often expected to be performed on videos before processing methods to either increase or decrease bitrate.

\subsection{Stereo Vision}
MathWorks (n.d.) defines stereo vision as a process of utilizing multiple 2D perspectives in order to extract information in 3D. In addition, most uses of stereo vision involve estimating an objects distance from an observer or camera. The 3D information is stated to be extracted with stereo pairs or pair of images through estimation of relative depth of points in a scene which are then represented through a stereo map that is made through the matching of the pair's corresponding points.


\section{Related Studies}
This section of the chapter presents related studies conducted by other researchers wherein the methodology and technologies used may serve as basis in the development of this special problem.

\subsection{Deep Learning Studies}

\noindent\textbf{\large Automated Detection and Classification of Road Anomalies in VANET Using Deep Learning} \\\\
In the study of Bibi et al. (2021) it was noted that identification of active road defects are critical in maintaining smooth and safe flow of traffic. Detection and subsequent repair of such defects in roads are crucial in keeping vehicles using such roads away from mechanical failures. The study also emphasized the growth in use of autonomous vehicles in research data gathering which is what the researchers utilized in data gathering procedures. With the presence of autonomous vehicles, this allowed the researchers to use a combination of sensors and deep neural networks in deploying artificial intelligence. The study aimed to allow autonomous vehicles to avoid critical road defects that can possibly lead to dangerous situations. Researchers used Resnet-18 and VGG-11 in automatic detection and classification of road defects. Researchers concluded that the trained model was able to perform better than other techniques for road defect detection. The study is able to provide the effectiveness of using deep learning models in training artificial intelligence for road defect detection and classification. However, the study lacks findings regarding the severity of detected defects and incorporation of pothole depth in their model which are both crucial in automating manual procedures of road surveying in the Philippines.


\noindent\textbf{\large Road Anomaly Detection through Deep Learning Approaches} \\\\
The study of \citeA{luo2020} aimed to utilize deep learning models in classifying road anomalies. The researchers used three deep learning approaches namely Convolutional Neural Network, Deep Feedforward Network, and Recurrent Neural Network from data collected through the sensors in the vehicle's suspension system. In comparing the performance of the three deep learning approaches, the researchers fixed some hyperparameters. Results revealed that the RNN model was the most stable among the three and in the case of the CNN and DFN models, the researchers suggested the use of wheel speed signals to ensure accuracy. And lastly, the researchers concluded that the RNN model was best due to high prediction performance with small set parameters. However, proper severity assessment through depth information was not stated to be utilized in any of the three approaches used in the study.

\noindent\textbf{\large Assessing Severity of Road Cracks Using Deep Learning-Based Segmentation and Detection} \\\\
In the study of \citeA{ha2022}, it was argued that the detection, classification, and severity assessment of road cracks should be automated due to the bottleneck it causes during the entire process of surveying. For the study, the researchers utilized SqueezNet, U-Net, and MobileNet-SSD models for crack classification and severity assessment. Furthermore, the researchers also employed separate U-nets for linear and area cracking cases. For crack detection, the researchers followed the process of pre-processing, detection, classification. During preprocessing images were smoothed out using image processing techniques. The researchers also utilized YOLOv5 object detection models for classification of pavement cracking wherein the YOLOv51 model recorded the highest accuracy. The researchers however stated images used for the study are only 2D images which may have allowed higher accuracy rates. Furthermore, the researchers suggest incorporating depth information in the models to further enhance results.

\noindent\textbf{\large Roadway pavement anomaly classification utilizing smartphones and artificial intelligence} \\\\
The study of \citeA{kyriakou2016} presented what is considered as a low-cost technology which was the use of Artificial Neural Networks in training a model for road anomaly detection from data gathered by smartphone sensors. The researchers were able to collect case study data using two-dimensional indicators of the smartphone’s roll and pitch values. In the study’s discussion, the data collected displayed some complexity due to acceleration and vehicle speed which lead to detected anomalies being not as conclusive as planned. The researchers also added that the plots are unable to show parameters that could verify the data’s correctness and accuracy. Despite the setbacks, the researchers still fed the data into the Artificial Neural Network that was expected to produce two outputs which were “no defect” and “defect.” The method still yielded above 90\% accuracy but due to the limited number of possible outcomes in the data processing the researchers still needed to test the methodology with larger data sets and roads with higher volumes of anomalies.

\subsection{Machine Learning Studies}

\noindent\textbf{\large Smartphones as Sensors for Road Surface Monitoring} \\\\
In their study, \citeA{sattar2018} noted the rise of sensing capabilities of smartphones which they utilized in monitoring road surface to detect and identify anomalies. The researchers considered different approaches in detecting road surface anomalies using smartphone sensors. One of which are threshold-based approaches which was determined to be quite difficult  due to several factors that are affecting the process of determining the interval length of a window function in spectral analysis. The researchers also utilized a machine learning approach adapted from another study. It was stated that k-means was used in classifying sensor data and in training the SVM algorithm. Due to the requirement of training a supervised algorithm using a labeled sample data was required before classifying data from sensors, the approach was considered to be impractical for real-time situations. In addition, \citeA{sattar2018} also noted various challenges when utilizing smartphones as sensors for data gathering such as sensors being dependent on the device’s placement and orientation, smoothness of captured data, and the speed of the vehicle it is being mounted on. Lastly, it was also concluded that the accuracy and performance of using smartphone sensors is challenging to compare due to the limited data sets and reported algorithms.

\noindent\textbf{\large Road Surface Quality Monitoring Using Machine Learning Algorithms} \\\\
The study of \citeA{singh2021} aimed to utilize machine learning algorithms in classifying road defects as well as predict their locations. Another implication of the study was to provide useful information to commuters and maintenance data for authorities regarding road conditions. The researchers gathered data using various methods such as smartphone GPS, gyroscopes, and accelerometers. \cite{singh2021} also argued that early existing road monitoring models are unable to predict locations of road defects and are dependent on fixed roads and static vehicle speed.  Neural and deep neural networks were utilized in the classification of anomalies which was concluded by the researchers to yield accurate results and are applicable on a larger scale of data. The study of \citeA{singh2021} can be considered as an effective method in gathering data about road conditions. However, it was stated in the study that relevant authorities will be provided with maintenance operation and there is no presence of any severity assessment in the study. This may cause confusion due to a lack of assessment on what is the road condition that will require extensive maintenance or repair.

\subsection{Computer Vision Studies}

\noindent\textbf{\large Stereo Vision Based Pothole Detection System for Improved Ride Quality} \\\\
In the study of \citeA{ramaiah2021} it was stated that stereo vision has been earning attention due to its reliable obstacle detection and recognition. Furthermore, the study also discussed that such technology would be useful in improving ride quality in automated vehicles by integrating it in a predictive suspension control system. The proposed study was to develop a novel stereo vision based pothole detection system which also calculates the depth accurately. However, the study focused on improving ride quality by using the 3D information from detected potholes in controlling the damping coefficient of the suspension system. Overall, the pothole detection system was able to achieve 84\% accuracy and is able to detect potholes that are deeper than 5 cm. The researchers concluded that such system can be utilized in commercial applications. However, it is also worth noting that despite the system being able to detect potholes and measure its depth, the overall severity of the pothole and road condition was not addressed.

\newpage
\section{Chapter Summary}
The reviewed literature involved various techniques and approaches in  road anomaly detection and classification. These approaches are discussed and summarized below along with their limitations and research gaps.
\begin{table}[h!]
	\centering
	\hspace{-2cm}
	\small 
	\begin{tabular}{|p{3cm}|p{3.5cm}|p{4.5cm}|p{4.5cm}|}
		\hline
		\textbf{Study} & \textbf{Technology/
			Techniques Used} & \textbf{Key Findings} & \textbf{Limitations} \\ \hline
		
		Automated Detection and Classification of Road Anomalies in VANET Using Deep Learning & Resnet-18 and VGG-11 & Trained model is able to provide the effectiveness of using deep learning models in training artificial intelligence for road defect detection and classification. & Lacks findings regarding the severity of detected defects. \\ \hline
		
		Smartphones as sensors for Road surface monitoring & Machine Learning, Smartphones & Approach was considered impractical for real-life applications. & Sensors are dependent on device's placement and orientation, smoothness of data, and speed of vehicle it is mounted on. Accuracy of results is difficult to compare. \\ \hline
		
		Road Anomaly Detection through Deep Learning Approaches & Convolutional Neural Network, Deep Feedforward Network, and Recurrent Neural Network & Identified that RNN was the best deep learning approach due to high prediction performance. & Data collection is considered too difficult and complicated to execute due to sensors being mounted on an integral part of the vehicle. \\ \hline
		
		Assessing Severity of Road Cracks Using Deep Learning-Based Segmentation and Detection & SqueezNet, U-Net, YOLOv5, and MobileNet-SSD models & YOLOv51 model recorded the highest accuracy. & Only 2D images are used for the study which may have allowed higher accuracy rates, and the study also lacked depth information. \\ \hline
		
		Stereo Vision Based Pothole Detection System for Improved Ride Quality & Pair of stereo images captured by a stereo camera & System was able to achieve 84\% accuracy and is able to detect potholes that are deeper than 5 cm. & Overall severity of the pothole and road condition was not addressed. \\ \hline
		
	\end{tabular}
	\caption{Comparison of Related Studies on Road Anomaly Detection using Deep Learning Techniques and Stereo Vision}
	\label{tab:comparison}
	\hspace{-3cm}
\end{table}