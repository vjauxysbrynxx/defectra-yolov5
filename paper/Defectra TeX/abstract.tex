%   Filename    : abstract.tex 
\begin{abstract}
Road surveying is a crucial part of the maintenance processes of roads in the Philippines that is carried out by the Department of Public Works and Highways. However, the current process of road surveying is time consuming which delays much needed maintenance operations. Existing studies involving automated pothole detection lack integration of the pothole's depth in assessing its severity which is essential for automating road  surveying procedures.  A system that incorporates estimated depth information in assessing pothole severity is developed in order to automate the manual process of depth measurement and severity assessment in road surveying. For depth estimation, stereo vision is favorable in this context as depth may be estimated through the disparity generated by a stereo pair. In obtaining a stereo view of the potholes, the StereoPi V2 is utilized along with some modifications that would make it eligible for outdoor use. After initially finding a non-linear relationship between the disparity and true depth, a curve fitting approach was utilized in order to relate disparity and ground-truth distance measurements. Linear regression analysis revealed a strong positive linear correlation between estimated and actual depth. Furthermore, the results displayed that the system with StereoPi V2 camera was able to effectively measure pothole depths mostly within 2 cm of their actual depth.

%  Do not put citations or quotes in the abract.

\begin{flushleft}
\begin{tabular}{lp{4.25in}}
\hspace{-0.5em}\textbf{Keywords:}\hspace{0.25em} & pothole, depth estimation, stereo vision, StereoPi V2\\
\end{tabular}
\end{flushleft}
\end{abstract}
